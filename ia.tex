\documentclass{report}
\usepackage[margin=1in]{geometry}
\usepackage{indentfirst}
%\addtolength{\topmargin}{-.5in}
\begin{document}
\begin{center}
\section*{The Relationship Between the K Value of a Spring and the Damping of a Simple Pendulum}
\textit{Suyog Soti}
\end{center}

\paragraph{Research Question:} How does the K value of a spring affect the damping constant of the pendulum that it is attached to?

\subsection*{Background and Personal Engagement}
% make this better
\indent I have always had a strong interest in Pendulums and Springs so I decided to combine these factors in the only way I knew how. I derived an equation for how the different K value of a spring will affect the damping constant of the pendulums, then I tested the my equation by attaching a spring to a pendulums and observing the forces through a sensor. 

\subsection*{Methodology}
\begin{enumerate}
\item Attach a spring to a force sensor and a mass as done in Figure one
\item After setting up the software of the sensor, measure the K value of the spring by measuring the displacement of the spring from before to after the weights
\item Put weights on the base of the ring stand so that the ring stand does not move when the pendulum moves
\item Bring the pendulum to a heigh with a reasonable angle(about 20 degrees) and drop the pendulum. The angle will not matter because the damping constant is being measured, not the velocity. Neither the period nor the velocity is reliant on the angle.
\item After letting go the pendulum, measure the force from the force sensor
\item Repeat steps 1 - 4 for 4 different springs
\end{enumerate}

\subsection*{Vairables}
\indent The independant variable in the lab was the spring because that was the only feasable way to change the spring constant. In changing the spring, the length of the spring was not kept constant because of two reasons. There were no springs that had the same length and different K values, and second, the purpose is to measure the affect on damping, not period. Damping is independant from the length of the pendulum. The dependant varaible in the lab was the force that the sensor finds between the spring and the hook of that the spring is attached to as shown in \textit{Figure 1} below. The mass was controlled for in the experiment by keeping it a constant, and the force sensor was not changed between each of the different springs.

\subsection*{Materials}
\begin{itemize}
\item Force sensor and its complementing software
\item Some springs of at least 12 cm in length
\item A max of about 300 grams
\item A ring stand set for the pendulum to hang of
\item Weights or in this experiment, books, so that the ring stand does not move
\item A balance beam with at least 0.1 gram accuracy minimum
%is the last one really needed?
\item Some tape so that the sensor does not swing off the ring stand
\end{itemize}

\subsection*{Diagram}
%attach a picture there

\subsection*{Equation Derivation}
%equation derivation here

\subsection*{Graphs}
%need graphs from raw data
%need graphs from the maximums
%need the graphs from just one of the max points



\end{document}